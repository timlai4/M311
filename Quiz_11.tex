\documentclass[addpoints]{exam}
\usepackage{amssymb}
\usepackage{amsmath}
\usepackage{graphicx}
\usepackage{makeidx}
\usepackage{hyperref}


\newcommand{\R}{\mathbf{R}}
\newcommand{\Z}{\mathbf{Z}}

\pagestyle{headandfoot}
\runningheadrule
\firstpageheader{Math M311}{Quiz 11}{December 3, 2019}
\runningheader{Math M311}
{Quiz 11, Page \thepage\ of \numpages}
{December 3, 2019}
\firstpagefooter{}{}{}
\runningfooter{}{}{}

\begin{document}
\printanswers
\begin{questions}
\question Show that 
\[
V = \iiint _ C dV = \pi R^2 h
\]
where $C$ is a cylinder with radius $R$ and height $h$. 
\begin{parts}
\part[5] First, orient the cylinder appropriately. Because we are computing volume, we can rotate and translate our cylinder however we'd like. Let's set it up so that the cylinder is sitting on the $xy$- plane ($z = 0$) and goes up to $z = h$. Now, project the cylinder to the $xy$-plane and draw a picture, with the $x$ and $y$-axes labeled, of the projection.
\part[5] Using your picture, write down an equation describing the projection. 
\part[5] Set up the double integral to compute the area of this projection. 
\part[5] Now, let's add in the height of the cylinder. The innermost integral is $\int_0^h 1 \, \ dz$. Wrap this into the double integral you got from the previous part and compute the iterated triple integral to attain $\pi R^2 h$. 
\end{parts}
\begin{solution}
Orient the cylinder so that its projection to the $xy$-plane is the unit circle centered at the origin, described by $r = R$. Then, 
\[
V = \int _0 ^{2\pi} \int_0^R \int_0^h r \, \ dz dr d \theta =  \int _0 ^{2\pi} \int_0^R h \, \ dr d \theta =  \frac{R^2 h}{2} \int _0 ^{2\pi} 1 \, \ d \theta = \pi R^2 h
\]
\end{solution}


\question[30] Show that 
\[
I = \int_{-\infty}^{\infty} e^{-x^2} \, \ dx = \sqrt{\pi}
\]
In particular, the \textit{normal distribution} is indeed a probability distribution. 

To begin, square both sides. On the LHS, we get:
\[
I^2 =  \int_{-\infty}^{\infty} e^{-x^2} \, \ dx  \int_{-\infty}^{\infty} e^{-y^2} \, \ dy =  \int_{-\infty}^{\infty}\int_{-\infty}^{\infty} e^{-(x^2 + y^2)} \, \ dx dy 
\]
Now, compute this double integral and you should get $\pi$. Conclude that $I = \sqrt{\pi}$. Why is it not equal to $- \sqrt{\pi}$?
\begin{solution}
\url{https://en.wikipedia.org/wiki/Gaussian_integral#By_polar_coordinates}
\end{solution}

\end{questions}
\end{document}