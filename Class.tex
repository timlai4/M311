\documentclass[xcolor=svgnames]{beamer}
\usepackage{graphics}
\usepackage{amsmath}
\usepackage[english]{babel}
\usetheme{Warsaw}
\useinnertheme{rounded}
\usefonttheme{serif}
\setbeamertemplate{navigation symbols}{}
\begin{document}
\title{M311 Calculus III Recitation}
	\author{Tim Lai }
	\institute{Indiana University}
%	\titlegraphic{\includegraphics[width=5cm]{Logo.jpg}}
	\date{Fall 2019}
\frame{\titlepage}
\begin{frame}
\frametitle{Projections}
\framesubtitle{Section 12.1}
\begin{center}
\includegraphics[width=8cm]{3D_vector_component.png}
\end{center}
\end{frame}
\begin{frame}
\frametitle{Vector Operations}
\framesubtitle{Section 12.2}
Suppose $a = \langle 2,3,0 \rangle$ and $b = \langle -1, 0,1 \rangle$. 
Then
\begin{align*}
 |a + 2b| &= |  \langle 2,3,0 \rangle + 2  \langle -1, 0,1 \rangle | \\
&= | \langle 2,3,0 \rangle +\langle -2, 0,2 \rangle | \\
& = | \langle 0,3,2 \rangle | \\
&= \sqrt{0^2 + 3^2 + 2^2} \\
&= \sqrt{13}
\end{align*}
\end{frame}
\begin{frame}
\frametitle{Dot Product}
\framesubtitle{Section 12.3}
Dot products are a special case of matrix multiplication.

Suppose $a = \langle 2,3,0 \rangle$ and $b = \langle -1, 0,1 \rangle$. Then
\[
 a \cdot b = 2(-1) + 3(0) + 0(1) = -2
\]
Compare to $(2,3,0) (-1,0,1)^T$ where these are now thought of as matrices and $T$ denotes transpose. 
\end{frame}
\begin{frame}
\frametitle{Dot Product}
\framesubtitle{Important properties and uses}
\begin{itemize}
	\item Dot product gives a computationally conducive way to get a handle on angles between vectors. 
	\item $a \cdot b = |a||b| \cos \theta$ where $\theta$ is the angle between the two vectors. 
	\item In particular, since $\cos \theta = 0$ if and only if $\theta = \pi / 2 + k \pi$, we can conclude that $a \perp b \Leftrightarrow a \cdot b = 0$. 
	\item Note: dot product takes two vectors and gives a number. 
	\item In the previous example, since the dot product was nonzero, we can conclusively say that the vectors were not orthogonal. 
\end{itemize}
\end{frame}
\begin{frame}
\frametitle{Cross Product}
\framesubtitle{Problem 20}
Computing unit vectors perpendicular to $j - k$ and $i + j$:
\[
	(j-k) \times (i + j) = i - j - k
\]
However, this vector is not a unit vector! We must normalize by dividing by the magnitude, in this case, $1/\sqrt{3}$. 
\end{frame}
\begin{frame}
\frametitle{Cross Product}
\framesubtitle{Problem 27}
Finding the area of a parallelogram given the corner points. 
\begin{enumerate}
\item Plot the vectors and draw in the parallelogram
\item Identify two vectors that determine the parallelogram
\item Compute the magnitude of the cross product. 
\end{enumerate}
\end{frame}
\begin{frame}
\frametitle{Cross Product}
\framesubtitle{Problem 29}
Finding a vector orthogonal to the plane given three points on the plane. 
\begin{enumerate}
\item The plane through $P,Q,R$ must contain the vectors formed by the points, such as $PQ$ and $PR$. 
\item Find a vector orthogonal to both of these vectors with the corss product.
\item This vector must then be orthogonal to the plane itself. 
\end{enumerate}
\end{frame}
\begin{frame}
\frametitle{Lines and Planes}
\framesubtitle{Problem 10}
Find equation of line containing $P_0 = (2,1,0)$ and direction perpendicular to $i+J$ and $j + k$. 
\begin{enumerate}
\item Compute a vector orthogonal to both 
\item Say the vector is $(a,b,c)$. Then the parametric equation is $x = 2 + at, y = 1 + bt, z = ct$. 
\end{enumerate}
\end{frame}
\begin{frame}
\frametitle{Lines and Planes}
\framesubtitle{Problem 17}
Line segment from $r_0 = \langle 6,-1,9 \rangle$ to $r_1 = \langle 7,6,0 \rangle $:
\begin{align*}
r(t) &= (1-t)r_0 + tr_1 \\
&= \langle 6, -1, 9 \rangle - t\langle 6, -1, 9 \rangle + t \langle 7, 6 , 0 \rangle \\
&= (1-t)\langle 6, -1, 9 \rangle  + t \langle 7, 6, 0 \rangle \\
&= \langle 6, -1, 9 \rangle + t \langle 1, 7, -9 \rangle
\end{align*}
\end{frame}
\begin{frame}
\frametitle{Lines and Planes}
\framesubtitle{Parallel (Problem 20)}
How do you determine whether two lines are parallel given their direction vectors? 
\begin{enumerate}
\item Say we have two lines with direction vectors $v_1$ and $v_2$. 
\item Determine whether there exists a constant such that $v_1 = c v_2$. 
\item If there is, then the two lines are parallel. If not, it may be possible that the lines intersect or are skew. 
\end{enumerate}


\end{frame}

\end{document}